\documentclass[a4paper]{ctexart}
\usepackage{amsmath}
\usepackage{graphicx}
\usepackage{hyperref}
\usepackage{float}
\usepackage{geometry}
\title{测量介质中的声速}
\author{陈启钰\,\,2300011447}
\date{\today}
\begin{document}
	\maketitle
	\tableofcontents
	\newpage
	\section{共振频率的测量}
	测量得到共振频率为
	\begin{align}
		f_0=39.75\mathrm{kHz}
	\end{align}
	\section{极值法测量声速}
	\begin{table}[H]
		\begin{center}
			\caption{正反向极值法测量结果}
			\begin{tabular}{c|cccccccccc}
				i&1&2&3&4&5&6&7&8&9&10\\
				\hline
				$x_i/\mathrm{mm}$&14.867&19.312&24.173&27.721&31.608&36.312&41.041&45.314&49.790&54.332\\
				\hline
				$U_{pp}/\mathrm{V}$&3.16&2.54&2.10&1.94&1.80&1.64&1.52&1.40&1.32&1.22\\
				\hline
				$x'_i/\mathrm{mm}$&63.117&58.779&54.210&49.696&45.056&40.611&36.072&31.400&26.862&23.029\\
				\hline
				$U'_{pp}/\mathrm{V}$&1.00&1.10&1.20&1.30&1.38&1.46&1.62&1.72&1.90&2.06
			\end{tabular}
		\end{center}
	\end{table}
	采用逐差法处理数据,令
	\begin{align}
		\Delta x_i=\begin{cases}
			x_{i+5}-x_{i}, 1\le i\le 5\\
			x'_{i}-x'_{i+5}, 6\le i\le 10
		\end{cases}
	\end{align}
	\begin{table}[H]
		\begin{center}
			\caption{逐差法结果}
			\begin{tabular}{c|cccccccccc}
				i&1&2&3&4&5&6&7&8&9&10\\
				\hline
				$\Delta x_i/\mathrm{mm}$&21.445&21.720&21.141&22.069&22.724&22.506&22.707&22.810&22.834&22.027				
			\end{tabular}
		\end{center}
	\end{table}
	\begin{align}
		\overline{\Delta x}=\frac{1}{10}\sum_{i=1}^{10}\Delta x_i=22.198\mathrm{mm}
	\end{align}
	\begin{align}
		\sigma_a&=\sqrt{\frac{1}{10\times9}\sum_{i=1}^{10}(\overline{\Delta x}-\Delta x_i)^2}=0.19\mathrm{mm}\\
		\sigma_b&=\frac{0.004}{\sqrt{3}}\mathrm{mm}=0.0024\mathrm{mm}
		\sigma=\sqrt{\sigma_a^2+\sigma_b^2}=0.20\mathrm{mm}
	\end{align}
	所以
	\begin{align}
		\Delta x=(22.20\pm0.20)\mathrm{mm}
	\end{align}
	\begin{align}
		\lambda=\frac{2}{5}\Delta x=(8.88\pm 0.08)\mathrm{mm}
	\end{align}
	共振频率的相对不确定度很小,相比波长可忽略,可认为共振频率的结果为精确值。
	计算声速
	\begin{align}
		v=\lambda f_0=(353.0\pm3.2)\mathrm{m/s}
	\end{align}
	\section{相位法测量声速}
	\begin{table}[H]
	\begin{center}
		\caption{正反向相位法测量结果}
		\begin{tabular}{c|cccccccccc}
			i&1&2&3&4&5&6&7&8&9&10\\
			\hline
			$x_i/\mathrm{mm}$&18.213&27.309&36.307&45.232&54.1799&63.060&71.721&80.387&89.242&97.966\\
			\hline
			$x'_i/\mathrm{mm}$&98.001&89.202&80.368&71.752&62.998&54.095&45.169&36.241&27.251&18.201
		\end{tabular}
	\end{center}
	\end{table}
	采用最小二乘法处理数据,对于正向测量结果,线性拟合可得
	\begin{align}
	\lambda=k=8.848\mathrm{mm},r=0.99997
	\end{align}
	\begin{align}
	\sigma_a&=\lambda\sqrt{\frac{1/r^2-1}{10-2}}=0.024\mathrm{mm}\\
	\sigma_b&=\frac{0.004\mathrm{mm}}{\sqrt{\sum_{i=1}^{10}(i-\overline{i})^2}}=0.0005\ll \sigma_a\\
	\sigma\approx\sigma_a=0.024\mathrm{mm}
	\end{align}
	所以
	\begin{align}
	\lambda_1=(8.848\pm0.024)\mathrm{mm}
	\end{align}
	正向测量得到的声速
	\begin{align}
	v_1=\lambda_1 f_0=(351.7\pm1.0)\mathrm{mm}
	\end{align}
	反向同理可得
	\begin{align}
	\lambda_2=(8.855\pm0.020)\mathrm{mm}
	\end{align}
	\begin{align}
	v_2=\lambda_2 f_0=(352.0\pm0.8)\mathrm{mm}
	\end{align}
	\section{气体参量法测量声速}
	温度(摄氏度)
	\begin{align}
		\theta=25.1^{\circ}\mathrm{C}
	\end{align}
	相对湿度
	\begin{align}
		H=52\%
	\end{align}
	大气压强
	\begin{align}
		p=761.10\mathrm{mmHg}=761.10\mathrm{mm}\times13.6\mathrm{g/cm^3}\times9.80\mathrm{m/s^2}=1.01\times10^5\mathrm{Pa}
	\end{align}
	查表知
	\begin{align}
		p_s=3167.6\mathrm{Pa}
	\end{align}
	\begin{align}
		p_w=Hp_s=1.6\times10^3\mathrm{Pa}
	\end{align}
	计算得到声速
	\begin{align}
		v=331.45\mathrm{m/s}\times\sqrt{\left(1+\frac{\theta}{T_0}\right)\left(1+\frac{0.3192p_w}{p}\right)}=348\mathrm{m/s}
	\end{align}
	这里的有效数字取法采用如下规则:如果是几个量之间的乘除法,则结果的有效数字位数与有效数字最少的物理量保持一致,所以$\frac{\theta}{T_0}$仅有两位有效数字,$\frac{0.3192p_w}{p}$有三位有效数字。本式中的加法有效数字位数也与最少的保持一致,因为这里的1是一个确定值、精确值,并不是一个测量量,认为它有无穷多位有效数字,所以$1+\frac{\theta}{T_0}$有两位有效数字,最后根号里式子的最后结果只有两位有效数字,即$1.1$。\\
	进行根号运算时,有效数字不一定只有两位,因为
	\begin{align}
		1.0^2=1.0,1.1^2=1.2
	\end{align}
	与$1.1$相差$0.1$,此时不得不多保留一位有效数字,故取$\sqrt{1.1}=1.05$,最后计算出的声速也有三位有效数字。
	\section{水中声速的测量}
	采用超声光栅的方法测量水中的声速,实验数据记录如下:
	超声光栅与墙面距离
	\begin{align}
		L=433.5\mathrm{cm}
	\end{align}
	超声波频率
	\begin{align}
		f=10.02\mathrm{MHz}
	\end{align}
	激光波长
	\begin{align}
		\Lambda=632.8\mathrm{nm}
	\end{align}
	\begin{table}[H]
		\begin{center}
			\caption{衍射条纹间距的测量}
			\begin{tabular}{c|ccccccc}
				$\Delta x/\mathrm{cm}$&1.80&1.77&1.80&1.80&1.80&1.75&1.77
			\end{tabular}
		\end{center}
	\end{table}
	平均值
	\begin{align}
		\overline{\Delta x}=1.78\mathrm{cm}
	\end{align}
	由
	\begin{align}
		\lambda=\frac{\Lambda}{\theta=\frac{\Lambda}{\overline{\Delta x}/L}=1.53\times10^{-4}\mathrm{m}
		\end{align}
		水中声速
		\begin{align}
			v_1=\lambda f=1.53\times10^3\mathrm{m/s}
		\end{align}
		这里只有三位有效数字,取决于$\Delta x$的有效数字位数。
		]
\end{document}